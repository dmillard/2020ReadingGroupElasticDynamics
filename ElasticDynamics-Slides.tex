\documentclass{article}
\usepackage[
  papersize={5in,3in},
  voffset=0.05in,
  headsep=0.1in,
  footskip=0.2in,
  lmargin=0.2in,
  rmargin=0.2in,
  tmargin=0.4in,
  bmargin=0.4in
]{geometry}
\usepackage[utf8]{inputenc}
\usepackage[english]{babel}
\usepackage{parskip}
\usepackage{fancyhdr}
\usepackage{mathtools}
\usepackage{amssymb}
\usepackage{graphicx}
\usepackage{tcolorbox}
\usepackage{physics}
\usepackage{algorithm2e}

\pagestyle{fancy}
\fancyhf{}
\lhead{\small{\textsc{\leftmark}}}
\lfoot{\small{\textsc{USC Robotic Embedded Systems Lab}}}
\rfoot{\small{\textsc{Slide \thepage}}}
\renewcommand{\footrulewidth}{\headrulewidth}

\title{Elastic Solid Simulation}
\author{Presented by David Millard}
\date{December 03, 2020}

\newcommand{\R}[1]{\ensuremath{\mathbb{R}^{#1}}}
\newcommand{\Rx}[2]{\ensuremath{\mathbb{R}^{#1 \times #2}}}
\newcommand{\tpose}[1]{\ensuremath{#1^\top}}
\newcommand{\vx}{\ensuremath{\vec{x}}}
\newcommand{\vxstar}{\ensuremath{\vec{x}_*}}
\newcommand{\vX}{\ensuremath{\vec{X}}}
\newcommand{\vXstar}{\ensuremath{\vec{X}_*}}
\newcommand{\vphi}{\ensuremath{\vec{\phi}}}
\newcommand{\vphiX}{\ensuremath{\vec{\phi}(\vX)}}
\newcommand{\vphiXstar}{\ensuremath{\vec{\phi}(\vXstar)}}
\newcommand{\F}{\ensuremath{\mathbf{F}}}
\newcommand{\I}{\ensuremath{\mathbf{I}}}
\newcommand{\Ephi}{\ensuremath{E[\vphi]}}
\newcommand{\PsiphiX}{\ensuremath{\Psi[\vphi; \vX]}}
\newcommand{\PsiphiXstar}{\ensuremath{\Psi[\vphi; \vXstar]}}
\newcommand{\PsiF}{\ensuremath{\Psi(\F)}}
\newcommand{\PsiFi}{\ensuremath{\Psi(\F_i)}}
\newcommand{\vf}{\ensuremath{\vec{f}}}
\newcommand{\fX}{\ensuremath{\vf(\vX)}}
\newcommand{\tauX}{\ensuremath{\vec{\tau}(\vX)}}
\newcommand{\fagg}{\ensuremath{\vec{f}_{\text{aggregate}}}}
\renewcommand{\P}{\ensuremath{\mathbf{P}}}
\newcommand{\PF}{\ensuremath{\P(\F)}}
\newcommand{\E}{\ensuremath{\mathbf{E}}}
\newcommand{\eps}{\ensuremath{\mathbf{\epsilon}}}
\newcommand{\GreenDefn}{\ensuremath{\E = \frac{1}{2}(\tpose{\F}\F - \I)}}
\newcommand{\bx}{\ensuremath{\mathbf{x}}}
\newcommand{\Ni}{\ensuremath{\mathcal{N}_i}}
\newcommand{\Ds}{\ensuremath{\mathbf{D}_s}}
\newcommand{\Dm}{\ensuremath{\mathbf{D}_m}}
\newcommand{\Ti}{\ensuremath{\mathcal{T}_i}}
\renewcommand{\H}{\ensuremath{\mathbf{H}}}

\begin{document}

\maketitle
\thispagestyle{empty}

\section{Overview}

\vspace{1em}
{\huge
  A tutorial on finite-element schemes for simulating elastic solids. Largely
  drawn from \cite{sifakis2012fem}
}

\pagebreak
\section{Definitions}

\subsection{Reference configuration}

$\Omega \subset \R{3}$ is the volumetric domain occupied by an object.

$\vX \in \Omega$ is a point in this undeformed domain.

\begin{tcolorbox}[title=Note]
  Capital letters $\vX$ are used for \emph{undeformed} points. Lower case
  letters $\vx$ will be used for \emph{deformed} points.
\end{tcolorbox}

\pagebreak
\subsection{Deformation function}
When an object deforms, each point $\vX$ is displaced to $\vx$.

This transform is the \emph{deformation function}.

$$\vphi : \R{3} \to \R{3}$$
$$\vx = \vphiX$$

\subsection{Deformation gradient tensor}

The \emph{deformation gradient} tensor $\F \in \Rx{3}{3}$ is the
Jacobian of $\vphi$.

\pagebreak
\begin{tcolorbox}[title=Examples]
  \begin{description}
    \item[Translation] $$\vphiX = \vX + \vec{t}, \quad \F = \pdv{\vphiX}{\vX} = \I$$
    \item[Uniform Scaling] $$\vphiX = \gamma\vX, \quad \F = \gamma\I$$
    \item[Rotation] $$\vphiX = \mathbf{R}_{\theta}\vX, \quad F = \mathbf{R}_{\theta}\I$$
  \end{description}
\end{tcolorbox}

\pagebreak
\subsection{Strain energy}

Elasticity: Deformation causes potential energy, called \emph{strain energy} $\Ephi$.

For \emph{hyperelastic} materials, energy depends only on \emph{final} deformed shape.

Locally, there is an \emph{energy density function} $\PsiphiX$ such that

$$\Ephi = \int_\Omega \PsiphiX d\vX$$

\pagebreak
We will approximate strain energy density $\PsiphiXstar$ at a particular point $\vXstar$.
\begin{align*}
  \vphiX & \approx \vphiXstar + \left. \pdv{\vphi}{\vX} \right\rvert_{\vXstar} (\vX - \vXstar)        \\
         & = \F(\vXstar)\vX + \underbrace{\vxstar - \F(\vXstar)\vXstar}_{\text{constant translation}}
\end{align*}

Translation alone doesn't effect strain energy, so $\PsiphiX \approx \PsiF$.

Different $\Psi$ will result in different material properties, we will revisit this.

\pagebreak
\subsection{Forces and traction}
What elastic forces are caused by deformation?

Continuous objects have a \emph{force density function} $\fX$
$$\fagg(A) = \int_{A \subset \Omega} \fX d\vX$$
Along the surface, this is replaced with \emph{traction} $\tauX$. This distinction
disappears with discretization.
$$\fagg(B) = \oint_{B \subset \partial \Omega} \tauX dS$$

\pagebreak
\subsection{First Piola-Kirchoff stress tensor}
A unifying concept for force and traction is the \emph{stress tensor}.

We consider the \emph{first Piola-Kirchoff stress tensor} $\P \in \Rx{3}{3}$ with properties

\begin{itemize}
  \item Traction at $\vX \in \partial\Omega$ with (reference) surface normal
        $\vec{N}$ is $$\tauX = -\P \cdot \vec{N}$$
  \item Internal force density is
        $$\fX = \div{\P(\vX)} \quad \text{i.e.} \quad f_i = \pdv{P_{i1}}{X_1} + \pdv{P_{i2}}{X_2} + \pdv{P_{i3}}{X_3} $$
\end{itemize}

\pagebreak
For hyperelastic matrials, $$\P(F) = \pdv{\PsiF}{\F}$$

\begin{tcolorbox}[title=Example]
  Consider $\PsiF = (k/2) \left\| \F - \I \right\|^2_{\F}$.
  $$\delta \PsiF = (k/2)\delta[(\F-\I):(\F-\I)] = k(\F-\I):\delta\F = \pdv{\Psi}{\F}:\delta\F$$
  and so
  $P(\F) = k(\F-\I)$
  and internal forces are $\vf = \div P = k\Delta\vphi$
\end{tcolorbox}

\pagebreak
\subsection{Lam\'e parameters $\mu$, $\lambda$}
Parameters used to parameterize materials (e.g.\ rubber vs steel).

Related to the Young's modulus $k$ of elasticity and Poisson ratio $\nu$ of
incompressibility.
$$\mu = \frac{k}{2(1 + \nu)}$$
$$\lambda = \frac{k\nu}{(1 + \nu)(1 - 2\nu)}$$

\pagebreak
\section{Material models}
A material can be described by Piola stress $\PF$ or $\PsiF$

\subsection{Strain measures}
$\F$ is unintuitive to reason about. Instead we use \emph{strain measures},
which

\begin{itemize}
  \item Quantify severity of deformation
  \item Discard irrelevant information (like rotations)
\end{itemize}

\pagebreak
\subsection{Green strain tensor}
The \emph{Green strain tensor} $\E \in \Rx{3}{3}$ is
$$\GreenDefn$$
and has several desirable properties
\begin{enumerate}
  \item At rest $\vphiX = \vX$, $\F = \I$, and $\E = 0$
  \item For rigid motion $\vphiX = \mathbf{R}\vX + \vec{t}$, $\F = \mathbf{R}$, and $\E = \tpose{\mathbf{R}}\mathbf{R} - \I = 0$
  \item Generally $\F = \mathbf{R}\mathbf{S}$ and $\E = \frac{1}{2}(\mathbf{S}^2 - I)$, discarding rotations
\end{enumerate}

\pagebreak
\subsection{Small strain tensor}
The \emph{small strain tensor} $\eps$ is a linearization of $\E$
$$\eps = \frac{1}{2}(\F + \tpose{\F}) - \I$$

\subsection{Linear elasticity}
The simplest constitutive model, which is valid for small relative motions
$$\PsiF = \mu\eps:\eps + \frac{\lambda}{2}\mathrm{tr}^2(\eps)$$
$$\P(\F) = 2\mu\eps + \lambda \mathrm{tr}(\eps)\I$$

\pagebreak
\subsection{St. Venant-Kirchhoff model}
Improve the approximation by using $\E$ instead of $\eps$
$$\PsiF = \mu\E:\E + \frac{\lambda}{2}\mathrm{tr}^2(\E)$$
$$\P(\F) = \F[2\mu\E + \lambda\mathrm{tr}(\E)\I]$$
An recall that the Green strain $\E$ is
$$\GreenDefn$$
Rotationally invariant, but poor resistance to forceful compression.

\pagebreak
\subsection{Other models}
Please refer to \cite{sifakis2012fem} for
\begin{itemize}
  \item Corotated linear elasticity
  \item Isotropic materials
  \item The Neohookean model
\end{itemize}

\pagebreak
\section{Discretization}

Compute $\vphiX$ on a finite number of nodes $\bx = \{\vX_i\}$.

Compute strain energy on small elements $e$, defined by the points in $\bx$.
$$\E[\phi] := \int_\Omega \PsiF d\vX$$
becomes
$$\sum_e \int_{\Omega_e} \Psi\left(\hat\F(\vX; \mathbf{x})\right)d\vX$$

\pagebreak
\subsection{Nodal forces}
Compute force $\vf_i$ per node $i$, summed from elements in its neighborhood $\Ni$
$$\vf_i(\bx) = \sum_{e\in\Ni} \vf_i^e(\bx), \quad \text{where} \quad \vf_i^e(\bx) = -\pdv{E^e(\bx)}{\vx_i}$$

\subsection{Linear tetrahedral elements}
Build elements from four points, and linearize $\phi$ per element.
$$
  \hat{\phi}(\vX) = \F\vX + \vec{b}
$$

\pagebreak
It is possible to recover $\F$ from the tetrahedron vertices.
\begin{align*}
  \begin{bmatrix}
    \vx_1 - \vx_4 \quad
    \vx_2 - \vx_4 \quad
    \vx_3 - \vx_4
  \end{bmatrix} & = \F\begin{bmatrix}
    \vX_1 - \vX_4 \quad
    \vX_2 - \vX_4 \quad
    \vX_3 - \vX_4
  \end{bmatrix} \\
  \Ds                       & =\F\Dm                         \\
  \F(x)                     & =\Ds(\bx)\Dm^{-1}
\end{align*}
Where $\Ds$ is called the \emph{deformed shape matrix} and $\Dm$ is called
the \emph{reference shape matrix}.

Since $\F$ is constant over the element, strain energy reduces to
$$
  E_i = \int_{\Ti} \PsiF d\vX = \PsiFi\int_{\Ti}d\vX = W \cdot \PsiFi
$$

\pagebreak
\subsection{Discretized algorithm}
\begin{algorithm}[H]
  \SetAlgoLined
  $f = 0$\;
  \ForEach{$\mathcal{T}_e = (i, j, k, l) \in \mathcal{M}$}{
  $\Ds = \begin{bmatrix}
      \vx_1 - \vx_4 \quad
      \vx_2 - \vx_4 \quad
      \vx_3 - \vx_4
    \end{bmatrix}$\;
  $\F = \Ds\Dm^e{}^{-1}$\;
  $\P = \P(F)$\;
  $\H = -W\P(\Dm^e{}^{-1}){}^\top$\;
  $
    \vf_i += \vec{h}_1,
    \vf_j += \vec{h}_2,
    \vf_k += \vec{h}_3
  $\;
  $\vf_l += -(\vec{h}_1, \vec{h}_2, \vec{h}_3)$\;
  }
\end{algorithm}

Where $
  \H = \begin{bmatrix}
    \vec{h_1} \quad
    \vec{h_2} \quad
    \vec{h_3}
  \end{bmatrix}
$ and $W = \frac{1}{6}\det(\Dm)$

\pagebreak
\bibliographystyle{apalike}
\bibliography{ElasticDynamics-Slides}

\end{document}